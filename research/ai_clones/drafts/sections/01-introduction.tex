The concept of AI clones and digital twins as extensions of self represents a significant frontier in human-computer interaction and psychological research. As artificial intelligence systems become increasingly sophisticated, the ability to create digital representations that mirror not just our physical characteristics but our cognitive and behavioral patterns has moved from science fiction to technological reality. This advancement raises fundamental questions about the nature of self-identity, agency, and the psychological implications of human-AI integration.

The development of AI clones---digital entities that learn from and emulate their human counterparts---has garnered significant attention across multiple disciplines. From healthcare applications where digital twins assist in personalized treatment planning \citep{kim2024healthcare} to social contexts where AI agents serve as extensions of human presence \citep{kawakami2020digital}, these technologies are reshaping our understanding of human-computer interaction and self-representation in digital spaces.

\subsection{Scope and Objectives}

This survey paper aims to comprehensively examine the current state of research regarding AI clones as extensions of self, with particular focus on three key areas:

\begin{enumerate}
\item Interface and System Designs: Analysis of technical frameworks and interaction paradigms that enable AI clone creation and operation
\item Psychological Implications: Investigation of cognitive, emotional, and behavioral effects of human-AI clone interaction
\item Agency and Self-Attribution: Examination of how humans perceive and attribute agency to AI clones, with special attention to the Japanese concept of 自己帰属 (self-attribution)
\end{enumerate}

The paper specifically addresses:
\begin{itemize}
\item Technical architectures and interface designs for AI clone systems
\item Psychological mechanisms underlying human-AI clone relationships
\item Cultural variations in self-attribution and agency perception
\item Ethical considerations in digital self-extension
\end{itemize}

\subsection{Methodology}

Our methodology encompasses a systematic review of academic literature across multiple disciplines and cultural contexts. We analyzed:
\begin{itemize}
\item Technical papers from major computing conferences and journals
\item Psychological research from both Western and Eastern perspectives
\item Cultural studies on digital identity and self-attribution
\item Case studies of implemented AI clone systems
\end{itemize}

The literature review focused on papers published between 2015-2024, with particular attention to recent developments (2020-2024) that reflect current technological capabilities and psychological understanding. Sources were selected based on:
\begin{itemize}
\item Relevance to AI clones and digital twins
\item Scientific rigor and methodology
\item Cultural diversity in perspective
\item Citation impact and scholarly influence
\end{itemize}

\subsection{Paper Organization}

The remainder of this paper is organized as follows:

Section 2 establishes the theoretical framework, defining key concepts and introducing fundamental theories of self-extension and agency. Section 3 examines interface and system designs, analyzing technical approaches to AI clone implementation. Section 4 explores psychological implications, focusing on identity formation and cognitive integration. Section 5 investigates agency and self-reference, with particular attention to cultural variations in self-attribution. Section 6 discusses current limitations and future directions, while Section 7 concludes with a synthesis of findings and research opportunities.

Throughout the paper, we maintain a cross-cultural perspective, particularly highlighting Japanese concepts and research contributions to provide a more comprehensive understanding of how different cultures approach and interpret AI clone technology and its implications for self-extension.
