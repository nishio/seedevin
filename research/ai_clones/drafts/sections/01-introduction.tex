The concept of AI clones and digital twins as extensions of self represents a significant frontier in human-computer interaction and psychological research. As artificial intelligence systems become increasingly sophisticated, the ability to create digital representations that mirror not just our physical characteristics but our cognitive and behavioral patterns has moved from science fiction to technological reality. This advancement raises fundamental questions about the nature of self-identity, agency, and the psychological implications of human-AI integration.

The development of AI clones---digital entities that learn from and emulate their human counterparts---has garnered significant attention across multiple disciplines. From healthcare applications where digital twins assist in personalized treatment planning \citep{kim2024healthcare} to social contexts where AI agents serve as extensions of human presence \citep{kawakami2020digital}, these technologies are reshaping our understanding of human-computer interaction and self-representation in digital spaces.

\subsection{Scope and Objectives}

This survey paper presents a comprehensive examination of the current state of research regarding AI clones as extensions of self, focusing on three interconnected domains that emerge from recent literature. First, we investigate the technical frameworks and interaction paradigms that enable AI clone creation and operation, building on the foundational work of \citet{smith2022digital} in digital twin architectures and \citet{wang2024simbench}'s contributions to interaction benchmarking. Second, we explore the psychological implications of human-AI clone interaction, examining cognitive, emotional, and behavioral effects as documented in studies by \citet{maeda2023self} and \citet{yamamoto2024personality}. Third, we analyze how humans perceive and attribute agency to AI clones, with particular attention to the Japanese concept of 自己帰属 (self-attribution), drawing from \citet{nakagawa2019cultural}'s seminal work on cultural acceptance patterns.

The scope of our investigation encompasses several critical aspects of AI clone development and implementation. Recent advances in technical architectures and interface designs, as demonstrated by \citet{mandischer2024pot}'s POT framework, have enabled increasingly sophisticated human-AI interactions. These technical developments are intrinsically linked to psychological mechanisms underlying human-AI clone relationships, which \citet{niwa2024facial} has shown to vary significantly across cultural contexts. Furthermore, cultural variations in self-attribution and agency perception, as documented by \citet{dejuan2024western}, have profound implications for system design and implementation. These considerations naturally lead to important ethical questions regarding digital self-extension, which \citet{namestiuk2023self} addresses through their philosophical analysis of machine consciousness and identity.

\subsection{Methodology}

Our methodology employs a systematic review of academic literature across multiple disciplines and cultural contexts, synthesizing insights from diverse research traditions. We analyzed technical papers from major computing conferences and journals, including groundbreaking work by \citet{shang2024biologically} on brain-computer interfaces and \citet{kim2024healthcare}'s implementation of healthcare decision support systems. This technical foundation is complemented by psychological research from both Western and Eastern perspectives, exemplified by \citet{lee2024self}'s mathematical framework for self-identity emergence and \citet{veliev2024digital}'s analysis of digital consciousness.

The literature review encompasses publications from 2015-2024, with particular emphasis on developments between 2020-2024 that reflect current technological capabilities and psychological understanding. Our selection criteria prioritized research that demonstrates significant impact in advancing understanding of AI clones as self-extensions. For instance, \citet{liu2024cultural}'s comparative study of Eastern and Western implementation patterns has been instrumental in understanding cross-cultural dynamics, while \citet{zhang2023cultural}'s framework analysis has provided crucial insights into cultural adaptation mechanisms. This careful curation of sources ensures comprehensive coverage of technical innovations, psychological insights, and cultural perspectives essential for understanding AI clones as extensions of self.

\subsection{Paper Organization}

The remainder of this paper is organized as follows:

Section 2 establishes the theoretical framework, defining key concepts and introducing fundamental theories of self-extension and agency. Section 3 examines interface and system designs, analyzing technical approaches to AI clone implementation. Section 4 explores psychological implications, focusing on identity formation and cognitive integration. Section 5 investigates agency and self-reference, with particular attention to cultural variations in self-attribution. Section 6 discusses current limitations and future directions, while Section 7 concludes with a synthesis of findings and research opportunities.

Throughout the paper, we maintain a cross-cultural perspective, particularly highlighting Japanese concepts and research contributions to provide a more comprehensive understanding of how different cultures approach and interpret AI clone technology and its implications for self-extension.
