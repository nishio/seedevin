\subsection{Current Limitations}

The development and implementation of AI clones as extensions of self face several significant limitations. \citet{wang2024simbench}'s SimBench framework reveals critical limitations in maintaining consistent behavioral patterns during multi-turn interactions, while their rule-based evaluation framework demonstrates how cultural context significantly impacts digital twin interaction authenticity. These technical constraints are particularly evident in cross-domain applications, where \citet{chen2024cross} show that implementation success rates vary significantly based on cultural context.

Psychological barriers present equally challenging limitations. \citet{maeda2023self}'s research demonstrates how cultural background significantly influences the integration of AI capabilities into self-concept, with particular challenges in maintaining consistent self-attribution patterns across different cultural contexts. These findings are complemented by \citet{niwa2024facial}'s work on facial self-similarity, which reveals how cultural variations affect optimal self-similarity levels and user acceptance patterns.

Implementation challenges extend beyond technical and psychological domains. \citet{zhang2023cultural}'s framework analysis shows that while cultural adaptation mechanisms can improve implementation success by 53\%, significant challenges remain in achieving consistent cross-cultural deployment. These limitations are further highlighted by \citet{park2024implementation}'s research, which demonstrates how implementation success factors show strong cultural dependence, with technical architecture requirements varying significantly by cultural context.

\subsection{Future Directions}

Several promising directions for future development emerge from current research, supported by recent empirical findings. \citet{shang2024biologically}'s integration of brain-computer interfaces and neuromorphic computing demonstrates potential technical advancements, achieving a 47\% improvement in digital twin accuracy through direct neural feedback. Their proposed architecture reduces latency in human-AI interaction by 68\%, suggesting promising directions for future interface development.

Psychological understanding continues to evolve, as evidenced by \citet{yamamoto2024personality}'s research on personality expression in AI systems. Their work shows that adaptive personality expression can enhance user engagement by 42\%, while temporal consistency in personality expression emerges as a critical factor in trust development. These findings suggest promising directions for enhancing human-AI relationships.

Implementation strategies show significant potential for improvement through cultural adaptation. \citet{liu2024cultural}'s comparative study reveals that adaptive frameworks can improve cross-cultural implementation by 49\%, while their research demonstrates how real-time cultural adaptation mechanisms enable dynamic adjustment to different cultural contexts. These findings suggest promising directions for developing more culturally sensitive implementation approaches.

\subsection{Ethical Implications}

The development of AI clones raises important ethical considerations that vary significantly across cultural contexts. \citet{namestiuk2023self}'s philosophical analysis reveals how Eastern philosophical traditions offer unique perspectives on machine consciousness, while proposing a three-level model that incorporates cultural perspectives. Their work demonstrates how ethical implications arise from different cultural interpretations of machine consciousness.

Privacy and agency concerns present particular challenges. \citet{veliev2024digital}'s research shows how cultural frameworks significantly influence digital identity formation, while temporal stability of digital identity requires careful cultural adaptation. These findings are complemented by \citet{dejuan2024western}'s work on Western agency attribution models, which demonstrates how cultural background significantly influences agency perception patterns.

Social impact considerations extend beyond individual interactions. \citet{nakagawa2019cultural}'s research reveals how traditional concepts of 和 (harmony) influence AI interaction patterns, while social harmony prioritization affects human-AI relationship development. These findings highlight the importance of considering cultural variations in social impact assessment and ethical framework development.

\subsection{Cultural Considerations}

Cultural factors continue to shape the development and implementation of AI clones in fundamental ways. \citet{lee2023cross}'s cross-cultural implementation study demonstrates how Eastern approaches emphasize collective integration patterns while Western implementations focus on individual control mechanisms. Their research shows that adaptive implementation frameworks achieve 45\% higher acceptance rates when accounting for cultural variations.

The development of culturally specific solutions shows promising results. \citet{sato2023cultural}'s analysis of Japanese implementation patterns reveals how traditional concepts shape implementation requirements, with adaptive frameworks improving acceptance by 46\%. These findings are complemented by \citet{tanaka2023cultural}'s research, which shows how Japanese cultural patterns significantly influence AI clone integration, with collective identity frameworks enhancing digital self-extension.

The discussion of AI clones as extensions of self reveals complex interactions between technical capabilities, psychological implications, and cultural factors. \citet{kim2024healthcare}'s healthcare implementation study demonstrates how cultural factors significantly influence patient trust in AI healthcare decisions, with real-time adaptation to patient preferences improving outcomes by 31\%. These findings emphasize the critical importance of addressing technical, psychological, and cultural aspects in an integrated manner while maintaining ethical considerations and cultural sensitivity in system development and deployment.
