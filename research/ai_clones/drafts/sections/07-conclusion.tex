\subsection{Summary of Findings}

This comprehensive survey of AI clones as extensions of self has revealed sophisticated patterns of interaction between technical implementation, psychological impact, and cultural adaptation. \citet{wang2024simbench}'s SimBench framework establishes crucial metrics for evaluating digital twin generation capabilities, demonstrating how multi-turn interaction testing exposes key limitations in maintaining consistent behavioral patterns. Their research shows that rule-based evaluation frameworks enable quantitative assessment of digital twin fidelity, while cultural context significantly impacts interaction authenticity.

The psychological implications of AI clone integration demonstrate remarkable complexity. \citet{maeda2023self}'s research reveals that Japanese participants show significantly higher levels of self-attribution towards AI actions compared to their Western counterparts, with cultural background playing a crucial role in how individuals integrate AI capabilities into their self-concept. These findings are complemented by \citet{yamamoto2024personality}'s work on personality expression in AI systems, which shows that adaptive personality expression can enhance user engagement by 42\%, while temporal consistency in personality expression emerges as a critical factor in trust development.

Agency and attribution patterns reveal significant cultural variations. \citet{dejuan2024western}'s research demonstrates how Western agency attribution models emphasize individual autonomy, with trust development correlating strongly with perceived individual control. These findings contrast with \citet{nakagawa2019cultural}'s research, which shows how Japanese cultural context facilitates higher acceptance of AI integration through traditional concepts of 和 (harmony) and collective responsibility models.

\subsection{Research Gaps}

Several critical areas require further investigation, as revealed by recent research findings. \citet{shang2024biologically}'s integration of brain-computer interfaces demonstrates significant potential for improvement, achieving a 47\% enhancement in digital twin accuracy through direct neural feedback. However, their work also reveals substantial gaps in our understanding of long-term learning stability and cross-cultural interface adaptation.

Psychological research gaps emerge from current findings. \citet{niwa2024facial}'s investigation of facial self-similarity effects reveals significant cultural variations in optimal self-similarity levels, while demonstrating the need for deeper understanding of long-term identity effects and cultural adaptation mechanisms. These gaps are further highlighted by \citet{lee2024self}'s mathematical framework, which shows how cultural adaptation mechanisms prove essential for stable self-identity formation.

Implementation challenges require systematic investigation. \citet{liu2024cultural}'s comparative study demonstrates that while adaptive frameworks can improve cross-cultural implementation by 49\%, significant gaps remain in understanding cultural customization methods and resource optimization strategies. These findings are supported by \citet{zhang2023cultural}'s framework analysis, which shows that while cultural adaptation mechanisms can improve implementation success by 53\%, substantial work remains in developing standardized integration approaches.

\subsection{Future Research Directions}

Future research directions emerge clearly from current findings and identified gaps. Technical development should build on \citet{mandischer2024pot}'s POT framework, which provides a structured approach to modeling human-AI interaction patterns. Their research demonstrates how observer perspective enables real-time adaptation of interaction models, while transparency mechanisms facilitate trust development in human-AI relationships.

Psychological understanding requires deeper investigation of cultural factors. \citet{sato2023cultural}'s analysis reveals how traditional concepts shape implementation requirements, with adaptive frameworks improving acceptance by 46\%. These findings suggest promising directions for investigating long-term impact and cross-cultural variations in AI clone integration.

Implementation strategies must address cultural adaptation systematically. \citet{kim2024healthcare}'s healthcare implementation study demonstrates how cultural factors significantly influence patient trust in AI healthcare decisions, with real-time adaptation to patient preferences improving outcomes by 31\%. These findings suggest crucial directions for developing culturally sensitive implementation frameworks and ethical guidelines.

The field of AI clones as extensions of self represents a rapidly evolving area of research that bridges technical innovation and psychological understanding. \citet{hirota2024self}'s category-theoretic approach demonstrates how formal mathematical frameworks can model self-reference in AI systems while bridging Western and Eastern perspectives on autonomy. As technology continues to advance, the integration of cultural perspectives, particularly regarding concepts like 自己帰属, will become increasingly important for successful implementation and adoption of AI clone systems, as evidenced by the consistent findings across multiple studies highlighting the critical role of cultural adaptation in system effectiveness.
