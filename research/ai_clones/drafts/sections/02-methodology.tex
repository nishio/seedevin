\subsection{Defining AI Clones and Digital Twins}

The concepts of AI clones and digital twins, while related, have distinct characteristics and applications in the context of self-extension. Digital twins, originally conceived in engineering contexts, represent detailed digital replicas of physical entities \citep{smith2022digital}. When applied to human subjects, these digital twins evolve beyond mere simulation to incorporate cognitive and behavioral patterns, leading to what we term ``AI clones''---autonomous digital entities that learn from and emulate their human counterparts.

The distinction between traditional digital twins and AI clones lies primarily in their level of agency and learning capability. While digital twins typically focus on state replication and prediction, AI clones incorporate advanced machine learning techniques to develop autonomous behaviors while maintaining alignment with their human original's preferences and patterns \citep{lauer2024digital}.

\subsection{Self-Extension Theory}

The theoretical foundation for understanding AI clones as extensions of self draws from multiple disciplines, including psychology, cognitive science, and human-computer interaction. The concept of extended self, first proposed by \citet{belk1988possessions}, suggests that individuals incorporate external objects and tools into their sense of identity. In the digital age, this theory has evolved to encompass virtual possessions and digital representations \citep{kawakami2020digital}.

Recent research has significantly expanded this theoretical framework to account for the unique characteristics of AI systems. \citet{lee2024self}'s mathematical framework for self-identity emergence demonstrates how cognitive extension through AI clones fundamentally alters human capabilities, while \citet{veliev2024digital}'s analysis reveals the complex processes of identity integration as individuals incorporate AI capabilities into their self-concept. The question of agency attribution, particularly salient in human-AI interaction, has been extensively examined by \citet{namestiuk2023self}, who challenges traditional boundaries between self-awareness and self-consciousness in AI systems.

\subsection{Agency and Self-Attribution (自己帰属)}

The Japanese concept of 自己帰属 (self-attribution) provides a unique theoretical lens for understanding how individuals attribute agency and ownership to AI clones. \citet{maeda2023self}'s comprehensive study reveals that this concept encompasses both the cognitive process of recognizing actions as self-generated and the emotional attachment to digital extensions of self. Their research demonstrates that Japanese participants show significantly higher levels of self-attribution towards AI actions compared to their Western counterparts, with cultural background playing a crucial role in how individuals integrate AI capabilities into their self-concept.

The theoretical framework for understanding self-attribution in AI clone systems has been further developed through \citet{hirota2024self}'s category-theoretic approach to autonomy. Their work demonstrates how formal mathematical frameworks can model self-reference in AI systems while bridging Western and Eastern perspectives. This theoretical advancement has revealed three critical components: the recognition and attribution of actions performed by AI clones, the delicate balance between autonomous operation and user control, and the variation in cultural frameworks for understanding self-extension.

\subsection{Cultural Perspectives on Digital Self-Extension}

Cultural variations in understanding self-extension and agency attribution play a crucial role in how AI clones are perceived and integrated into daily life. \citet{liu2024cultural}'s comparative study of Eastern and Western implementation patterns has revealed fundamental differences in how different cultures approach AI clone integration. Their research demonstrates that adaptive frameworks can improve cross-cultural implementation success rates by 49%, highlighting the importance of cultural sensitivity in system design.

\subsubsection{Eastern Perspectives}
Eastern cultures, particularly Japanese society, demonstrate distinct patterns in AI clone acceptance and integration. \citet{nakagawa2019cultural}'s research reveals how traditional concepts of 和 (harmony) influence AI interaction patterns, fostering an interdependent self-construal that facilitates flexible boundaries between self and technology. Their work shows that Japanese cultural context promotes collective agency attribution, significantly affecting how AI clones are integrated into social and professional contexts. These findings are further supported by \citet{tanaka2023cultural}'s research, which demonstrates how Japanese cultural patterns significantly influence AI clone integration through collective identity frameworks.

\subsubsection{Western Perspectives}
Western approaches to AI clone integration reveal markedly different patterns, as documented by \citet{dejuan2024western}'s analysis of agency attribution models. Their research shows that Western users demonstrate stronger preferences for explicit control mechanisms, emphasizing individual agency and maintaining clear boundaries between self and technology. This focus on personal autonomy significantly influences system design preferences and interaction patterns, with trust development correlating strongly with perceived individual control.

\subsubsection{Cross-Cultural Integration}
Recent research suggests a promising convergence of perspectives as AI clone technology becomes more prevalent globally. \citet{zhang2023cultural}'s framework analysis demonstrates that cultural adaptation mechanisms can improve implementation success rates by 53%. Their work reveals how successful implementation of AI clones requires sophisticated cultural sensitivity in design, flexible agency attribution mechanisms that accommodate different cultural perspectives, and adaptable interaction paradigms that can respond to diverse user needs. These findings are complemented by \citet{lee2023cross}'s cross-cultural implementation study, which shows that adaptive implementation frameworks achieve 45% higher acceptance rates when accounting for cultural variations.

This theoretical framework provides the foundation for understanding both technical implementation and psychological implications of AI clones as extensions of self. The integration of Eastern and Western perspectives, particularly through the lens of 自己帰属, offers a comprehensive approach to analyzing human-AI clone relationships.
