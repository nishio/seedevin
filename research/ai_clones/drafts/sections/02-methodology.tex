\subsection{Defining AI Clones and Digital Twins}

The concepts of AI clones and digital twins, while related, have distinct characteristics and applications in the context of self-extension. Digital twins, originally conceived in engineering contexts, represent detailed digital replicas of physical entities \citep{smith2022digital}. When applied to human subjects, these digital twins evolve beyond mere simulation to incorporate cognitive and behavioral patterns, leading to what we term ``AI clones''---autonomous digital entities that learn from and emulate their human counterparts.

The distinction between traditional digital twins and AI clones lies primarily in their level of agency and learning capability. While digital twins typically focus on state replication and prediction, AI clones incorporate advanced machine learning techniques to develop autonomous behaviors while maintaining alignment with their human original's preferences and patterns \citep{lauer2024digital}.

\subsection{Self-Extension Theory}

The theoretical foundation for understanding AI clones as extensions of self draws from multiple disciplines, including psychology, cognitive science, and human-computer interaction. The concept of extended self, first proposed by \citet{belk1988possessions}, suggests that individuals incorporate external objects and tools into their sense of identity. In the digital age, this theory has evolved to encompass virtual possessions and digital representations \citep{kawakami2020digital}.

Recent research has expanded this framework to account for the unique characteristics of AI systems:
\begin{itemize}
\item Cognitive Extension: How AI clones extend human cognitive capabilities
\item Identity Integration: The process of incorporating AI capabilities into self-concept
\item Agency Attribution: How humans attribute agency to their digital extensions
\end{itemize}

\subsection{Agency and Self-Attribution (自己帰属)}

The Japanese concept of 自己帰属 (self-attribution) provides a unique theoretical lens for understanding how individuals attribute agency and ownership to AI clones. This concept encompasses both the cognitive process of recognizing actions as self-generated and the emotional attachment to digital extensions of self \citep{maeda2023self}.

Key theoretical components include:
\begin{enumerate}
\item Action Recognition: How individuals recognize and attribute actions performed by their AI clones
\item Control Perception: The balance between autonomous operation and user control
\item Cultural Variations: Different cultural frameworks for understanding self-extension
\end{enumerate}

\subsection{Cultural Perspectives on Digital Self-Extension}

Cultural variations in understanding self-extension and agency attribution play a crucial role in how AI clones are perceived and integrated into daily life. Research has identified significant differences between Eastern and Western perspectives:

\subsubsection{Eastern Perspectives}
Eastern cultures, particularly Japanese society, often emphasize:
\begin{itemize}
\item Interdependent self-construal
\item Flexible boundaries between self and technology
\item Collective agency attribution
\end{itemize}
These perspectives influence how AI clones are integrated into social and professional contexts \citep{nakagawa2019cultural}.

\subsubsection{Western Perspectives}
Western approaches typically focus on:
\begin{itemize}
\item Individual agency
\item Clear boundaries between self and technology
\item Personal autonomy
\end{itemize}
These cultural differences manifest in system design preferences and interaction patterns \citep{dejuan2024western}.

\subsubsection{Cross-Cultural Integration}
Recent research suggests a convergence of perspectives as AI clone technology becomes more prevalent globally. Studies indicate that successful implementation of AI clones requires:
\begin{itemize}
\item Cultural sensitivity in design
\item Flexible agency attribution mechanisms
\item Adaptable interaction paradigms
\end{itemize}

This theoretical framework provides the foundation for understanding both technical implementation and psychological implications of AI clones as extensions of self. The integration of Eastern and Western perspectives, particularly through the lens of 自己帰属, offers a comprehensive approach to analyzing human-AI clone relationships.
